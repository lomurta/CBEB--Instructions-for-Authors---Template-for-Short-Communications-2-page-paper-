\documentclass[nouppercase]{ifmbe}


\title{Tutorial Maker: an easy way to produce teaching material for 3D Slicer to promote open imaging science.}

\affiliation{Institution/Department, Affiliation, City, Country }{FIRSTAFF}
\affiliation{Institution/Department, Affiliation, City, Country }{SECONDAFF}

\author{A.B. Firstauthor}{FIRSTAFF}
\author{C. Coauthor}{SECONDAFF}
\author{D.E. Othercoauthor}{FIRSTAFF}

\begin{document}

\maketitle

\begin{abstract}
3D Slicer is an open-source software designed for visualization, processing, segmentation, registration, and analysis of medical, biomedical, and other 3D images and meshes. It also facilitates planning and navigating image-guided procedures. Tutorial Maker is a semi-automated tool that assists in creating educational content on 3D Slicer, specifically tutorials. It simplifies the process of producing teaching material, making it easier for radiology instructors, experienced users, and developers to share knowledge and guide others through various 3D Slicer functionalities. Whether it is creating step-by-step instructions, video demonstrations, or interactive guides, a tutorial maker streamlines the creation process and promotes effective learning.
\end{abstract}

\begin{keywords}
This section should contain a maximum of 5 words separated by a semicolon (e.g., cardiac death; ECG; ionic channels; arrhythmias).
\end{keywords}

\section{Introduction}

This document is formatted so that you may use this as a template for your manuscript. Divide the manuscript according to the standard scientific practise with main headings into main sections: Introduction, Materials and Methods, Results and discussion, Conclusions, Acknowledgments, and References as presented in this document. 
The short communication can be prepared in MS Word and submitted as a *.pdf file. The templates can be downloaded from the CBEB website: www.cbeb.org.br.


\section{Materials and Methods}

\subsection{Figures and Tables}

Figures may span the whole page width (see Figure~\ref{fig:fig1}) or use only a part of it with the text flowing on the left-hand side. Figures should be placed within the manuscript in the appropriate locations. Captions appear below figures but above tables, as in Figure~\ref{fig:fig1} and Table~\ref{tab:tab1}.

\textit{{Third level headings}}
If needed, the subsections may be divided into third-level sections with third-level headings to clarify the work description. Do not number the third-level headings.

\begin{figure}[ht]
      \centering
          \includegraphics[width=1\columnwidth]{figures/logo sbeb.png}
      \caption{Figure legends should contain enough information to understand the illustration without referring to the text but should be concise. The reader should get a reasonable understanding of the article by only looking at the figures and reading their legends.}
      \label{fig:fig1}
\end{figure}

\begin{table}[bt]
\centering
\caption{This table is placed after the Section Break at the end of text.}
\label{tab:tab1}
\begin{tabular}{|p{1.5cm}|p{1.5cm}|p{1.5cm}|p{1.5cm}|}
\hline
 &  &  & \\\hline
 &  &  & \\\hline
 &  &  & \\\hline
 &  &  & \\\hline
 &  &  & \\\hline
 &  &  & \\\hline
 % Please only put a hline at the end of the table
\end{tabular}
\end{table}

\subsection{Equations:} Number equations consecutively with the number placed in parenthesis to the extreme right on the Equation’s line. Refer to equations as Equation~(\ref{Equation:eq1}). At the beginning of a sentence, the word Equation should be spelt out. In the mathematical equations, all symbols used and their dimensions should be defined. As an example, the following Equation is given: (The Equation is written with the Equation tools)

\begin{equation}\label{Equation:eq1}
E = mc^2
\end{equation}

where
E = energy of the particle at rest [J],
m = mass of the particle [kg],
c = speed of light [m/s].

\subsection{References:} References should be sorted alphabetically according to the first author. If the first author appears more than once, these references shall be placed in chronological order. 
The reference formats for a journal article, a book and conference proceedings are illustrated in the References section such as~\cite{IFMBE}, \cite{smith99}, \cite{lock03}, \cite{south01}. The format is compact and does not include titles for articles. 

\section{Results and Discussion}
This paper makes many important points.

\section{Conclusions}
The conclusions are stated in this section, and they should be concise.

\section*{Acknowledgements}
The authors would like to express their gratitude to…

\bibliography{example}
(3 to 5 Refs recommended)
\end{document}
